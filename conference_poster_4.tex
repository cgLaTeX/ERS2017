\documentclass[landscape,a0paper,fontscale=0.285]{baposter} % Adjust the font scale/size here

\usepackage{graphicx} % Required for including images
\graphicspath{{figures/}} % Directory in which figures are stored

\usepackage{amsmath} % For typesetting math
\usepackage{amssymb} % Adds new symbols to be used in math mode
\usepackage{physics}
\usepackage{booktabs} % Top and bottom rules for tables
\usepackage{enumitem} % Used to reduce itemize/enumerate spacing
\usepackage{palatino} % Use the Palatino font
\usepackage[font=small,labelfont=bf]{caption} % Required for specifying captions to tables and figures

\usepackage{multicol} % Required for multiple columns
\setlength{\columnsep}{1.5em} % Slightly increase the space between columns
\setlength{\columnseprule}{0mm} % No horizontal rule between columns

\usepackage{tikz} % Required for flow chart
\usetikzlibrary{shapes,arrows} % Tikz libraries required for the flow chart in the template

\newcommand{\compresslist}{ % Define a command to reduce spacing within itemize/enumerate environments, this is used right after \begin{itemize} or \begin{enumerate}
\setlength{\itemsep}{1pt}
\setlength{\parskip}{0pt}
\setlength{\parsep}{0pt}
}

\definecolor{spartangreen}{RGB}{24,69,59} % Defines the color used for content box headers

\begin{document}

\begin{poster}
{
headerborder=closed, % Adds a border around the header of content boxes
colspacing=1em, % Column spacing
bgColorOne=white, % Background color for the gradient on the left side of the poster
bgColorTwo=white, % Background color for the gradient on the right side of the poster
borderColor=spartangreen, % Border color
headerColorOne=black, % Background color for the header in the content boxes (left side)
headerColorTwo=spartangreen, % Background color for the header in the content boxes (right side)
headerFontColor=white, % Text color for the header text in the content boxes
boxColorOne=white, % Background color of the content boxes
textborder=roundedleft, % Format of the border around content boxes, can be: none, bars, coils, triangles, rectangle, rounded, roundedsmall, roundedright or faded
eyecatcher=true, % Set to false for ignoring the left logo in the title and move the title left
headerheight=0.1\textheight, % Height of the header
headershape=roundedright, % Specify the rounded corner in the content box headers, can be: rectangle, small-rounded, roundedright, roundedleft or rounded
headerfont=\Large\bf\textsc, % Large, bold and sans serif font in the headers of content boxes
%textfont={\setlength{\parindent}{1.5em}}, % Uncomment for paragraph indentation
linewidth=2pt % Width of the border lines around content boxes
}
%----------------------------------------------------------------------------------------
%	TITLE SECTION 
%----------------------------------------------------------------------------------------
%
{\includegraphics[height=4em]{logo.png}} % First university/lab logo on the left
{\bf\textsc{Quantum Electrodynamics: A Microscopic Formulation of the Maxwell-Bloch equations}\vspace{0.2em}} % Poster title
{\textsc{Connor Glosser, Jack Hamel, B. Shanker, and Carlo Piermarocchi}} % Author names and institution
{\includegraphics[height=4em]{logo.png}} % Second university/lab logo on the right

%----------------------------------------------------------------------------------------
%	OBJECTIVES
%----------------------------------------------------------------------------------------

\headerbox{Objectives}{name=objectives,column=1,row=0}{



\begin{enumerate}\compresslist
\item Employ fast and accurate methods for calculating the dynamics of partial order and disordered quantum dot systems.
\item Calculate classical electric field radiation using semiclassical methods.
\item Investigate lasing effects, coupled Rabi oscillations, nonlinear propagation, and other optical phenomena seen in the system's dynamics. 
\end{enumerate}

\vspace{0.3em} % When there are two boxes, some whitespace may need to be added if the one on the right has more content
}

%----------------------------------------------------------------------------------------
%	INTRODUCTION
%----------------------------------------------------------------------------------------

\headerbox{Introduction}{name=introduction,column=0,row=0,bottomaligned=objectives}{

The dynamics of ordered quantum dot systems can be described using analytic methods similar to those used in electronic transport theories.  Those techniques do not work well when attempting to describe systems with disorder or partial order.  We are using an integral approach to the full-wave framework that describes the dynamics of these more realistic systems, which allows direct communication between quantum dots in the system.
}

%----------------------------------------------------------------------------------------
%	RESULTS 1
%----------------------------------------------------------------------------------------

\headerbox{Results 1}{name=results,column=2,span=2,row=0,bottomaligned=objectives}{

\begin{multicols}{2}
\vspace{1em}
\begin{center}
\includegraphics[width=0.8\linewidth]{placeholder}
\captionof{figure}{Figure caption}
\end{center}

Aliquam auctor, metus id ultrices porta, risus enim cursus sapien, quis iaculis sapien tortor sed odio. Mauris ante orci, euismod vitae tincidunt eu, porta ut neque. Aenean sapien est, viverra vel lacinia nec, venenatis eu nulla. Maecenas ut nunc nibh, et tempus libero. Aenean vitae risus ante. Pellentesque condimentum dui. Etiam sagittis purus non tellus tempor volutpat. Donec et dui non massa tristique adipiscing.
\end{multicols}

%------------------------------------------------

\begin{multicols}{2}
\vspace{1em}
Sed fringilla tempus hendrerit. Vestibulum ante ipsum primis in faucibus orci luctus et ultrices posuere cubilia Curae; Etiam ut elit sit amet metus lobortis consequat sit amet in libero. Lorem ipsum dolor sit amet, consectetur adipiscing elit. Phasellus vel sem magna. Nunc at convallis urna. isus ante. Pellentesque condimentum dui. Etiam sagittis purus non tellus tempor volutpat. Donec et dui non massa tristique adipiscing. Quisque vestibulum eros eu.

\begin{center}
\includegraphics[width=0.8\linewidth]{placeholder}
\captionof{figure}{Figure caption}
\end{center}

\end{multicols}
}

%----------------------------------------------------------------------------------------
%	REFERENCES
%----------------------------------------------------------------------------------------

\headerbox{References}{name=references,column=0,above=bottom}{

\renewcommand{\section}[2]{\vskip 0.05em} % Get rid of the default "References" section title
\nocite{*} % Insert publications even if they are not cited in the poster
\small{ % Reduce the font size in this block
\bibliographystyle{unsrt}
\bibliography{sample} % Use sample.bib as the bibliography file
}}

%----------------------------------------------------------------------------------------
%	Conclusion
%----------------------------------------------------------------------------------------

\headerbox{Conclusion}{name=futureresearch,column=1,span=2,aligned=references,above=bottom}{ % This block is as tall as the references block

\begin{multicols}{2}
Integer sed lectus vel mauris euismod suscipit. Praesent a est a est ultricies pellentesque. Donec tincidunt, nunc in feugiat varius, lectus lectus auctor lorem, egestas molestie risus erat ut nibh.

Maecenas viverra ligula a risus blandit vel tincidunt est adipiscing. Suspendisse mollis iaculis sem, in \emph{imperdiet} orci porta vitae. Quisque id dui sed ante sollicitudin sagittis.
\end{multicols}
}

%----------------------------------------------------------------------------------------
%	CONTACT INFORMATION
%----------------------------------------------------------------------------------------

\headerbox{Contact Information}{name=contact,column=3,aligned=references,above=bottom}{ % This block is as tall as the references block

\begin{description}\compresslist
\item[Web] www.university.edu/smithlab
\item[Email] john@smith.com
\item[Phone] +1 (000) 111 1111
\end{description}
}

%----------------------------------------------------------------------------------------
%	something
%----------------------------------------------------------------------------------------

\headerbox{something goes here}{name=conclusion,column=1,span=1,row=0,below=results,above=references}{

\begin{multicols}{2}

\tikzstyle{decision} = [diamond, draw, fill=blue!20, text width=4.5em, text badly centered, node distance=2cm, inner sep=0pt]
\tikzstyle{block} = [rectangle, draw, fill=blue!20, text width=5em, text centered, rounded corners, minimum height=4em]
\tikzstyle{line} = [draw, -latex']
\tikzstyle{cloud} = [draw, ellipse, fill=red!20, node distance=3cm, minimum height=2em]

\begin{tikzpicture}[node distance = 2cm, auto]
\node [block] (init) {Initialize Model};
\node [cloud, left of=init] (Start) {Start};
\node [cloud, right of=init] (Start2) {Start Two};
\node [block, below of=init] (init2) {Initialize Two};
\node [decision, below of=init2] (End) {End};
\path [line] (init) -- (init2);
\path [line] (init2) -- (End);
\path [line, dashed] (Start) -- (init);
\path [line, dashed] (Start2) -- (init);
\path [line, dashed] (Start2) |- (init2);
\end{tikzpicture}

%------------------------------------------------

\begin{itemize}\compresslist
\item Pellentesque eget orci eros. Fusce ultricies, tellus et pellentesque fringilla, ante massa luctus libero, quis tristique purus urna nec nibh. Phasellus fermentum rutrum elementum. Nam quis justo lectus.
\item Vestibulum sem ante, hendrerit a gravida ac, blandit quis magna.
\item Donec sem metus, facilisis at condimentum eget, vehicula ut massa. Morbi consequat, diam sed convallis tincidunt, arcu nunc.
\item Nunc at convallis urna. isus ante. Pellentesque condimentum dui. Etiam sagittis purus non tellus tempor volutpat. Donec et dui non massa tristique adipiscing.
\end{itemize}

\end{multicols}
}

%----------------------------------------------------------------------------------------
%	Problem Statement
%----------------------------------------------------------------------------------------

\headerbox{Problem Statement}{name=method,column=0,below=objectives,bottomaligned=conclusion}{ % This block's bottom aligns with the bottom of the conclusion block

We wish to calculate the dynamics of a system of N two-level quantum dots with transition frequency, $\omega_0$, among radiation with frequency $\omega_L$, where $\omega_0 \approx \omega_L$.  We start by formulating a density matrix with the Louiville-von Neumann equation
\begin{equation}
  i \hbar \pdv{\hat{\rho}}{t} = \commutator{\hat{\mathcal{H}}(t)}{\hat{\rho}} - i \hbar \hat{\mathcal{D}}\qty[\hat{\rho}]
  \label{eq:liouville}
\end{equation}
where $\hat{\mathcal{H}}$ and $\hat{\mathcal{D}}$ are the Hamiltonian and Dissipator operators of the system.
\begin{subequations}
  \begin{align}
    \hat{\mathcal{H}}(t) &\equiv \mqty(0 & \hbar \chi(t) \\ \hbar \chi^*(t) & \hbar \omega_0) \label{eq:hamiltonian} \\
    \hat{\mathcal{D}}[\hat{\rho}] &\equiv \mqty((\rho_{00} - 1)/T_1 & \rho_{01}/T_2 \\ \rho_{10}/T_2 & \rho_{11}/T_1) \label{eq:dissipator}
  \end{align}
\end{subequations}
where $\chi(t) \equiv \vb{d} \cdot \vb{E}/\hbar$. Therefore, in order to calculate these matrices, the electric field must be calculated using the electric dyadic Green's function
\begin{gather}
    \vb{E}(\vb{r}, t) = \vb{E}_0(\vb{r}, t) - \frac{\mu_0}{4\pi} \int
      \qty(I - 3\bar{\vb{r}}\bar{\vb{r}}) \frac{c^2 \vb{P}(\vb{r}', t_R)}{\abs{\vb{r} - \vb{r}'}^3} + \nonumber \\
      \qty(I - 3\bar{\vb{r}}\bar{\vb{r}}) \frac{c \dot{\vb{P}}(\vb{r}', t_R)}{\abs{\vb{r} - \vb{r}'}^2} +
      \qty(I -  \bar{\vb{r}}\bar{\vb{r}}) \frac{\ddot{\vb{P}}(\vb{r}', t_R)}{\abs{\vb{r} - \vb{r}'}} \dd[3]{\vb{r}'}
  \label{eq:efield}
\end{gather}
}

%----------------------------------------------------------------------------------------
%	RESULTS 2
%----------------------------------------------------------------------------------------

\headerbox{Results 2}{name=results2,column=2,span=2,below=objectives,bottomaligned=conclusion}{ % This block's bottom aligns with the bottom of the conclusion block

Donec faucibus purus at tortor egestas eu fermentum dolor facilisis. Maecenas tempor dui eu neque fringilla rutrum. Mauris \emph{lobortis} nisl accumsan.

\begin{center}
\begin{tabular}{l l l}
\toprule
\textbf{Treatments} & \textbf{Response 1} & \textbf{Response 2}\\
\midrule
Treatment 1 & 0.0003262 & 0.562 \\
Treatment 2 & 0.0015681 & 0.910 \\
Treatment 3 & 0.0009271 & 0.296 \\
\bottomrule
\end{tabular}
\captionof{table}{Table caption}
\end{center}

Nulla ut porttitor enim. Suspendisse venenatis dui eget eros gravida tempor. Mauris feugiat elit et augue placerat ultrices. Morbi accumsan enim nec tortor consectetur non commodo.

\begin{center}
\begin{tabular}{l l l}
\toprule
\textbf{Treatments} & \textbf{Response 1} & \textbf{Response 2}\\
\midrule
Treatment 1 & 0.0003262 & 0.562 \\
Treatment 2 & 0.0015681 & 0.910 \\
Treatment 3 & 0.0009271 & 0.296 \\
\bottomrule
\end{tabular}
\captionof{table}{Table caption}
\end{center}
}

%----------------------------------------------------------------------------------------

\end{poster}

\end{document}